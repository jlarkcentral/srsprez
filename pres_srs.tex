\documentclass[12pt]{beamer}

\usetheme{Amsterdam}
\usepackage{thumbpdf}
\usepackage{wasysym}
\usepackage{ucs}
\usepackage[french]{babel}
\usepackage[utf8]{inputenc}
\usepackage{pgf,pgfarrows,pgfnodes,pgfautomata,pgfheaps,pgfshade}
\usepackage{verbatim}
\usepackage{listings}
\lstset{language=Java,
                basicstyle=\footnotesize\ttfamily,
                keywordstyle=\footnotesize\color{blue}\ttfamily,
}

\pdfinfo
{
  /Title       (Spaced Repetition System)
}


\title{Spaced Repetition System}
%\date{September 6th 2006}

\begin{document}

\frame{\titlepage}





%%%%%%%%%%%%%%%%%%%%%%%%%%%%%%%%%%%%%%%%%
%%%%%%%%%% Content starts here %%%%%%%%%%
%%%%%%%%%%%%%%%%%%%%%%%%%%%%%%%%%%%%%%%%%



\section{Intro}

\begin{frame}
  \frametitle{SRS en 1 schema}
  (boites de plus en plus grandes)
\end{frame}

\section{Pourquoi}
\begin{frame}
  \frametitle{Pourquoi utiliser le SRS ?}
  \begin{block}{Problème == le cerveau}
    \begin{itemize}
      \item Apprendre 1000 définitions prend un temps fou
      \item L'apprentissage passif ne garantit pas une mémoire à long terme
    \end{itemize}
  \end{block}
  
  \begin{block}{Solution == SRS}
    \begin{itemize}
      \item La répétition en apprentissage actif assure (à priori) des souvenirs plus solides
      \item L'espacement permet de réduire le nombre de sollicitations du cerveau
    \end{itemize}
  \end{block}
\end{frame}

\section{Comment}
\begin{frame}
  \frametitle{Principes}
  \begin{block}{Apprendre != comprendre}
   On suppose que ce qui est à retenir est déjà compris (ou n'a pas besoin d'être compris)
  \end{block}

  \begin{block}{Questions simples}
    \begin{itemize}
      \item On ne demande qu'une information à la fois
      \item La question est concise
    \end{itemize}
  \end{block}
\end{frame}

\begin{frame}
  \frametitle{Idées reçues}
  \begin{itemize}
    \item \textit{Memory is infinite}
    \item \textit{The more you repeat the better}
    \item \textit{Learning by doing is the best}
  \end{itemize}
  
  \alert{$\rightarrow$ Il faut répéter régulièrement et intelligemment}
\end{frame}


\section{Algo}
\begin{frame}
  \frametitle{Algorithme SR}
  Il existe plusieurs versions de l'algorithme (SM-O à SM-8...+) de plus
  en plus opimisées. La démarche reste la même:
  
  Pour un ensemble de questions, initialiser l'intervalle de répétition de chaque question au minimum
  Puis pour chaque question posée,
  \begin{itemize}
   \item Recueillir la difficulté de la réponse selon l'utilisateur (sur une échelle de 1 à n...)
   \item Mettre à jour l'intervalle de répétition en fonction de la difficulté, du temps de réponse...
  \end{itemize}

\end{frame}


\begin{frame}[fragile]
  \frametitle{Implémentation du SM2}
  \begin{lstlisting}
  public static void calcuateInterval(Card card) {
      if (card.getEFactor() < 3) {
          card.setCount(1);
      }
      int count = card.getCount();
      int interval = 1;
      if (count == 2) {
          interval = 6;
      } else if (count > 2) {
         interval =  Math.round(card.getInterval() 
		      * card.getEFactor());
      }
      card.setInterval(interval);
  }
  \end{lstlisting}
\end{frame}

\end{document}

